\documentclass[UTF8,12pt]{ctexart}
\CTEXsetup[format={\Large\bfseries}]{section}
\usepackage{graphicx}
\usepackage{subfigure}
\usepackage{float}
\usepackage{amsmath}
\usepackage[a4paper,top=2cm,bottom=1.5cm,left=2cm,right=2.5cm]{geometry}
\usepackage{caption}
\usepackage{enumerate}
\usepackage[nottoc]{tocbibind}
\usepackage{indentfirst}
\usepackage{ctex}
\usepackage{fancyhdr}
\usepackage{bm}
\usepackage{siunitx}
\usepackage{amssymb}
\usepackage{draftwatermark}
\usepackage{everypage}
\usepackage{multirow}
\usepackage{algorithm}
\usepackage{algorithmicx}
\usepackage{algpseudocode}
\usepackage[colorlinks,linkcolor=blue]{hyperref}
\usepackage{pgfplots}
\usepackage{marvosym}
\usepackage{color}
\usepackage{listings}
\lstset{
    %行号
    numbers=left,
      %背景框
      framexleftmargin=10mm,
      frame=none,
     %背景色
     %backgroundcolor=\color[rgb]{1,1,0.76},
     backgroundcolor=\color[RGB]{255,245,244},
     %样式
     %keywordstyle=\bf\color{blue},
     %identifierstyle=\bf,
     %numberstyle=\color[RGB]{0,192,192},
     %commentstyle=\it\color[RGB]{0,96,96},   
     %stringstyle=\rmfamily\slshape\color[RGB]{128,0,0},
     %显示空格
     %showstringspaces=false
}
\def\equationautorefname{}
\floatname{algorithm}{Algorithm: }
\renewcommand{\algorithmicrequire}{\textbf{Input:}}
\renewcommand{\algorithmicensure}{\textbf{Output:}}
\def\figureautorefname{图}
\def\tableautorefname{表}%
\SetWatermarkText{}
\SetWatermarkLightness{0.92}
\SetWatermarkScale{0.49}
\pagestyle{fancy}
\numberwithin{equation}{section}
\CTEXoptions[today=old]
%\captionsetup[figure]{name={Fig.}}
\title{\heiti \textbf{EI313\ 课程大作业二}}
\author{\kaishu 杨凯翔 519030910240, \href{mailto: flying_feixiang@sjtu.edu.cn}{\underline{flying\_feixiang@sjtu.edu.cn}}}
\bibliographystyle{plain}
\begin{document}
\maketitle

\thispagestyle{fancy}
\fancyhead{}
\lhead{\bfseries SJTU EI313, Autumn, 2021}
\rhead{\bfseries 工科创作业2}
\lfoot{}
\cfoot{\zihao{5}\copyright \ 2021 Kaixiang Yang. All Rights Reserved.}
\rfoot{\thepage}
\renewcommand{\headrulewidth}{0.4pt}
\renewcommand{\footrulewidth}{0.4pt}

\zihao{5}
\heiti
\section{作业内容}
\begin{enumerate}
    \item 从官网下载 QEMU5.2.0 并编译
    \item 利用 QEMU 创建两个虚拟机并设置为 tap 网络模式,虚拟网卡类型分别为:e1000 和 virtio-net
    \item 通过 VNC viewer 或者 SSH 连接虚拟机
    \item 比较两种有着不同种虚拟网卡的虚拟机和主机之间的网络性能
\end{enumerate}
\indent 由于 tap 网络模式通过网桥将虚拟网卡与主机网卡相连,所以虚拟机都是通过主机的网卡访问外网,所以我们需要测试两种虚拟机和主机间的网络性能。
为了便日后学习与研究,我安装了ubuntu双系统,并在 ubuntu 上完成作业内容。
\section{安装双系统}
\begin{enumerate}
    \item 选择 ubuntu-20.10 来安装,从官网下载镜像 (\url{https://ubuntu.com/download/desktop}) 并用 Rufus 工具制作系统盘如\autoref{fig1}所示。
    \item 划分磁盘空间,因为我的电脑有两块盘(C盘为固态,D盘为机械),所以采用如\autoref{fig2}所示的分区方式,在C盘划分的空间作为EFI引导区(开机时可以选择进入哪个系统),在D盘分配256G的空间。
    \item 从U盘启动安装系统,自定义安装在已划分好的空间(freespace)/home, /boot 等分区可以按需选择(这里我只分了swap交换区和根目录),并选择C盘的那部分空间作为引导,安装系统。
\end{enumerate}
\begin{figure}[H]
    \centering
    \includegraphics[scale = 0.3]{picture/0.png}
    \caption{制作系统盘}
    \label{fig1}
\end{figure}
\begin{figure}[H]
    \centering
    \includegraphics[scale = 0.5]{picture/0-1.png}
    \caption{划分磁盘空间}
    \label{fig2}
\end{figure}
\section{下载并编译安装QEMU}
首先通过官网下载所需版本的 QEMU (\url{https://download.qemu.org/qemu-5.2.0.tar.xz}) 并通过命令行解压。
\begin{lstlisting}
$ xz -d qemu-5.2.0.tar.xz
$ tar -xvf qemu-5.2.0.tar
\end{lstlisting}
\begin{figure}[H]
    \centering
    \subfigure[pic1.]{
    \begin{minipage}[t]{0.5\linewidth}
    \centering
    \includegraphics[width=2.5in]{picture/1.png}
    %\caption{fig1}
    \end{minipage}%
    }%
    \subfigure[pic2.]{
    \begin{minipage}[t]{0.5\linewidth}
    \centering
    \includegraphics[width=2.5in]{picture/2.png}
    %\caption{fig2}
    \end{minipage}%
    }%
    \caption{Download}
\end{figure}
\subsection{安装一些工具及依赖包}
鉴于 ubuntu 系统是新安装的,很多软件等都没有安装,下面首先安装本次作业内容所需要的。
\begin{lstlisting}
$ sudo apt-get update
$ sudo apt-get install gcc
$ sudo apt-get install make
$ sudo apt-get install python3
\end{lstlisting}
在尝试配置、编译 QEMU 时,会报告一些错误,针对这些错误安装相应的内容:

\noindent \textbf{\color[RGB]{255,30,30}{ERROR: Cannot find Ninja}}
\begin{lstlisting}
$ sudo apt-get install ninja-build
$ sudo apt-get install build-essential
\end{lstlisting}
\textbf{\color[RGB]{255,30,30}{ERROR: glib-2.48 gthread-2.0 is required to compile QEMU}}
\begin{lstlisting}
$ apt-cache search glib2
$ sudo apt-get install libglib2.0
\end{lstlisting}
\textbf{\color[RGB]{255,30,30}{Unable to locate package libpixman-1-0-dev}}
\begin{lstlisting}
$ sudo apt-get install libpixman-1-dev
\end{lstlisting}
此时编译安装 qemu 安装虚拟机后会出现提示:\textbf{VNC Server running on 127.0.0.1:5900...}

这是由缺少SDL库或者在配置时没有使能SDL所导致的默认的vnc viewer无法形成可视化界面,所以这里我们预先安装:
\begin{lstlisting}
$ sudo apt-get install libsdl2-dev
\end{lstlisting}
\subsection{编译安装QEMU}
首先使用 ./configure 配置编译选项,用于生成 Makefile ,可以通过命令 \textbf{\color[RGB]{100,100,235}{./configure --help}} 查看可用选项。
注意到编译时可以指定 QEMU 支持的架构,其余的可以不编译,这样可以减少编译时间。
\begin{lstlisting}[breaklines=true]
$ ./configure --enable-kvm --enable-debug --enable-vnc --enable-werror --enable-sdl --target-list="x86_64-softmmu"
\end{lstlisting}
这里指定了x86\_64架构,同时显式地使其支持 \textbf{KVM} 和 \textbf{vnc},编译时将 warning 当做 error 处理,以及开启 SDL 支持。

配置完成后,在输出信息中确认已经有 SDL 支持。下面直接编译安装即可。
\begin{lstlisting}
$ make -j8
$ make install
\end{lstlisting}
\section{创建虚拟机并配置网络}
\subsection{检查KVM是否可用}
可以采用如下命令检查:
\begin{lstlisting}
$ grep -E 'vmx|svm' /proc/cpuinfo
$ lsmod | grep kvm
\end{lstlisting}
有输出即可。比如第二个命令的输出类似于:
\begin{lstlisting}
kvm_intel           *****   0
kvm                 *****   1   kvm_intel
\end{lstlisting}
\subsection{创建虚拟机并安装操作系统}
首先创建虚拟机镜像文件:
\begin{lstlisting}
$ qemu-img create -f qcow2 ubuntu.img 30G
$ qemu-img create -f qcow2 ubunntu3.img 30G
\end{lstlisting}
然后启动虚拟机并安装操作系统,为加以区分,虚拟机的操作系统使用 ubuntu-20.04.3, 同样从官网下载镜像(\url{https://ubuntu.com/download/desktop})
\begin{lstlisting}[breaklines=true]
$ qemu-system-x86_64 -m 4096 ubuntu.img -enable-kvm -cdrom ./ubuntu-20.04.3-desktop-amd64.iso
$ qemu-system-x86_64 -m 4096 ubuntu3.img -enable-kvm -cdrom ./ubuntu-20.04.3-desktop-amd64.iso
\end{lstlisting}
安装虚拟机过程如图 4所示:
\begin{figure}[H]
    \centering
    \subfigure[pic1.]{
    \begin{minipage}[t]{0.25\linewidth}
    \centering
    \includegraphics[width=1.5in]{picture/3.png}
    \label{4-1}
    %\caption{fig1}
    \end{minipage}%
    }%
    \subfigure[pic2.]{
    \begin{minipage}[t]{0.25\linewidth}
    \centering
    \includegraphics[width=1.5in]{picture/4.png}
    %\caption{fig2}
    \end{minipage}%
    }%
    \subfigure[pic3.]{
    \begin{minipage}[t]{0.25\linewidth}
    \centering
    \includegraphics[width=1.5in]{picture/5.png}
    %\caption{fig2}
    \end{minipage}
    }%
    \subfigure[pic4.]{
    \begin{minipage}[t]{0.25\linewidth}
    \centering
    \includegraphics[width=1.5in]{picture/8.png}
    %\caption{fig2}
    \end{minipage}
    }%

    \subfigure[pic5.]{
    \begin{minipage}[t]{0.25\linewidth}
    \centering
    \includegraphics[width=1.5in]{picture/9.png}
    %\caption{fig1}
    \end{minipage}%
    }%
    \subfigure[pic6.]{
    \begin{minipage}[t]{0.25\linewidth}
    \centering
    \includegraphics[width=1.5in]{picture/11.png}
    %\caption{fig2}
    \end{minipage}%
    }%
    \subfigure[pic7.]{
    \begin{minipage}[t]{0.25\linewidth}
    \centering
    \includegraphics[width=1.5in]{picture/12.png}
    %\caption{fig2}
    \end{minipage}
    }%
    \subfigure[pic8.]{
    \begin{minipage}[t]{0.25\linewidth}
    \centering
    \includegraphics[width=1.5in]{picture/13.png}
    %\caption{fig2}
    \end{minipage}
    }
    \label{fig3}
    \caption{虚拟机安装过程}
\end{figure}%/
\subsection{配置tap网络}
学习并了解了相关知识(\url{https://wiki.qemu.org/Documentation/Networking}),tap 网络模式是在主机中创建网桥和tap设备,然后使网桥连接主机的网卡和tap设备,再通过指定的tap设备和虚拟机的虚拟网卡相连接,达到访问外部网络的目的(tap设备是在数据链路层模拟的)这种连接模式会为虚拟机分配一个独立的局域网IP地址。

这里会用到一些网络相关的工具,如下安装:
\begin{lstlisting}
$ sudo apt-get install net-tools
$ sudo apt-get install ethtool
\end{lstlisting}
使用 \textbf{\color[RGB]{100,100,235}{ifconig}} 查看网络设备如\autoref{5-1}所示
\begin{figure}[H]
    \centering
    \subfigure[pic1.]{
        \begin{minipage}[t]{0.5\linewidth}
        \centering
        \includegraphics[width=2.8in]{picture/15.png}
        \label{5-1}
        %\caption{fig1}
        \end{minipage}%
        }%
        \subfigure[pic2.]{
        \begin{minipage}[t]{0.5\linewidth}
        \centering
        \includegraphics[width=2.8in]{picture/16.png}
        \label{5-2}
        %\caption{fig2}
        \end{minipage}%
        }%
    \label{ifconfig}
    \caption{查看主机网络设备}
\end{figure}
下面在主机创建网桥和两个tap设备,先安装所需工具
\begin{lstlisting}
$ sudo apt-get install bridge-utils
$ sudo apt-get install iproute2
\end{lstlisting}
然后创建网桥、tap设备、网桥分别与主机有线网卡和tap设备相连、并为网桥分配 IP
\begin{lstlisting}
$ modprobe tun tap #make sure module has built-in
$ ip link add br0 type bridge
$ ip tuntap add dev tap0 mode tap
$ ip tuntap add dev tap1 mode tap
$ ip link set dev enp4s0 master br0
$ ip link set dev tap0 master br0
$ ip link set dev tap1 master br0
$ ip link set dev br0 up
$ dhclient br0
$ ifconfig tap0 up
$ ifconfig tap1 up
\end{lstlisting}
再次通过 \textbf{\color[RGB]{100,100,235}{ifconfig}} 查看网络设备如\autoref{5-2}所示,通过命令 \textbf{\color[RGB]{100,100,235}{brctl show}} 查看网桥连接状态:
\begin{figure}[H]
    \centering
    \includegraphics[scale=0.6]{picture/25.png}
    \label{fig6}
    \caption{网桥连接情况}
\end{figure}
\subsection{启动虚拟机并配置网络}
执行如下命令启动虚拟机:
\begin{lstlisting}[breaklines=true]
$ qemu-system-x86_64 -m 4096 ubuntu.img -netdev tap,id=mynet0,ifname=tap0,script=no,downscript=no -device e1000,netdev=mynet0,mac=52:55:00:d1:55:01 -enable-kvm
$ qemu-system-x86_64 -m 4096 ubuntu3.img -netdev tap,id=mynet0,ifname=tap1,script=no,downscript=no -device virtio-net-pci,netdev=mynet0,mac=52:55:00:d1:55:02 -enable-kvm
\end{lstlisting}
主机的机器名字为 \textbf{ykx-ubuntu},两个虚拟机的名字分别为 \textbf{ykx-vm1, ykx-vm3},用来区分。

在两个虚拟机中分别安装工具 \textbf{net-tools} 和 \textbf{ethtool},分别查看网络设备和网卡信息,如图\autoref{7-1}、\autoref{7-2}、\autoref{7-3}、\autoref{7-4}所示。
\begin{figure}[H]
    \centering
    \subfigure[vm1]{
        \begin{minipage}[t]{0.5\linewidth}
        \centering
        \includegraphics[width=2.8in]{picture/17.png}
        \label{7-1}
        %\caption{fig1}
        \end{minipage}%
        }%
        \subfigure[vm3]{
        \begin{minipage}[t]{0.5\linewidth}
        \centering
        \includegraphics[width=2.8in]{picture/21.png}
        \label{7-2}
        %\caption{fig2}
        \end{minipage}%
        }%

        \subfigure[vm1-e1000]{
            \begin{minipage}[t]{0.5\linewidth}
            \centering
            \includegraphics[width=2.8in]{picture/18.png}
            \label{7-3}
            %\caption{fig1}
            \end{minipage}%
            }%
            \subfigure[vm3-virtio]{
            \begin{minipage}[t]{0.5\linewidth}
            \centering
            \includegraphics[width=2.8in]{picture/22.png}
            \label{7-4}
            %\caption{fig2}
            \end{minipage}%
            }%
    \label{vmifconfig}
    \caption{查看虚拟机网络设备和网卡信息}
\end{figure}
\subsection{通过ssh连接虚拟机或者主机} \label{section4.5}
分别在主机和两个虚拟机中安装 \textbf{ssh},下图展示主机连接 \textbf{ykx-vm3} 和 \textbf{ykx-vm3} 连接主机。
\begin{figure}[H]
    \centering
    \subfigure[主机连接虚拟机]{
        \begin{minipage}[t]{0.5\linewidth}
        \centering
        \includegraphics[width=2.8in]{picture/27.png}
        \label{8-1}
        %\caption{fig1}
        \end{minipage}%
        }%
        \subfigure[虚拟机连接主机]{
        \begin{minipage}[t]{0.5\linewidth}
        \centering
        \includegraphics[width=2.8in]{picture/28.png}
        \label{8-2}
        %\caption{fig2}
        \end{minipage}%
        }%
    \label{ssh}
    \caption{ssh连接}
\end{figure}
\section{网络性能测试和比较}
\subsection{连接外网}
在 \autoref{section4.5} 中已经成功通过 ssh 从主机连接虚拟机以及从虚拟机连接主机。再测试虚拟机是否可以访问外网
\begin{figure}[H]
    \centering
    \subfigure[vm1]{
        \begin{minipage}[t]{0.5\linewidth}
        \centering
        \includegraphics[width=2.8in]{picture/20.png}
        \label{9-1}
        %\caption{fig1}
        \end{minipage}%
        }%
        \subfigure[vm3]{
        \begin{minipage}[t]{0.5\linewidth}
        \centering
        \includegraphics[width=2.8in]{picture/24.png}
        \label{9-2}
        %\caption{fig2}
        \end{minipage}%
        }%
    \label{ssh}
    \caption{虚拟机连接外网}
\end{figure}
\subsection{测试虚拟机到主机的网络性能,并进行比较}
因为虚拟机访问外网都是通过主机的网卡,所以为比较两种虚拟网卡设备的性能,我们分别测试两个虚拟机到主机的速度来比较。
这里使用 iperf 工具来测试网络性能,在主机和虚拟机中分别安装 iperf 工具,(\textbf{\color[RGB]{100,100,235}{sudo apt-get install iperf3}})。使用方法参考 \url{https://www.cnblogs.com/saneri/p/11169926.html}

首先以主机作为服务端监听一个窗口:
\begin{lstlisting}
$ iperf3 -s -p 8080
\end{lstlisting}

然后分别在两个虚拟机上测试与主机的连接:
\begin{lstlisting}
$ iperf3 -c 192.168.1.131 -p 8080 -i 1
\end{lstlisting}
测试结果如下图所示:
\begin{figure}[H]
    \centering
    \subfigure[vm1]{
        \begin{minipage}[t]{0.33\linewidth}
        \centering
        \includegraphics[width=2in]{picture/19.png}
        \label{10-1}
        %\caption{fig1}
        \end{minipage}%
        }%
        \subfigure[vm3]{
        \begin{minipage}[t]{0.33\linewidth}
        \centering
        \includegraphics[width=2in]{picture/23.png}
        \label{10-2}
        %\caption{fig2}
        \end{minipage}%
        }%
        \subfigure[server]{
        \begin{minipage}[t]{0.33\linewidth}
        \centering
        \includegraphics[width=2in]{picture/26.png}
        \label{10-2}
        %\caption{fig2}
        \end{minipage}%
        }%
    \label{iperf3}
    \caption{比较网络性能}
\end{figure}
总结如\autoref{tab1}:
\begin{table}[H]
    \centering
    \caption{比较结果}
    \begin{tabular}{c|c}
    \hline
    & Bitrate \\
    \hline
    e1000 & 1.27 Gbits/sec \\
    virtio-net & 23.0 Gbits/sec  \\
    \hline       
    \end{tabular}
    \label{tab1}
\end{table}
即 virtio-net 的性能远好于 e1000
\section{补充}
\subsection{关于与无线网卡的网桥连接}
参考 \url{https://wiki.qemu.org/Documentation/Networking/NAT},在这部分内容中介绍了建立网桥连接无向网卡的方法,这里提供了一个脚本来代替命令行参数设定来启动虚拟机。
\subsection{virt-manager尝试}
virt-manager 是一个虚拟机管理器,提供了图形化界面来创建和管理虚拟机,通过它可以便捷地创建虚拟机。但是在使用的时候,它会默认使用 qemu-system-i386 的模拟器来创建虚拟机,这个默认设置需要通过在软件内改写 XML
来设置。另一方面,在用其创建虚拟机时对网络的设置时,如果想要使用在命令行创建的网桥,也要修改 XML 来设置,但在我的尝试下,它无法识别我在主机创建的tap设备。

\section{心得体会}
在这项大作业中,经历过诸多试错过程,最终完成了全部的内容。在此期间了解了更多的网络设备虚拟化的知识和 qemu 的使用,对 linux 系统的使用更加得心应手 (双系统重装了一次)。另一方面,在配置虚拟机的网络时,对虚拟机连接网络的各种方式都作以了解和学习,同时联系计算机网络课程中学到的知识,对这项作业帮助很多。


\end{document} 