\documentclass[UTF8,12pt]{ctexart}
\CTEXsetup[format={\Large\bfseries}]{section}
\usepackage{graphicx}
\usepackage{subfigure}
\usepackage{float}
\usepackage{amsmath}
\usepackage[a4paper,top=3cm,bottom=2cm,left=2cm,right=2.5cm]{geometry}
\usepackage{caption}
\usepackage{enumerate}
\usepackage[nottoc]{tocbibind}
\usepackage{indentfirst}
\usepackage{ctex}
\usepackage{fancyhdr}
\usepackage{bm}
\usepackage{siunitx}
\usepackage{amssymb}
\usepackage{draftwatermark}
\usepackage{everypage}
\usepackage{multirow}
\usepackage{algorithm}
\usepackage{algorithmicx}
\usepackage{algpseudocode}
\usepackage[colorlinks,linkcolor=blue]{hyperref}
\usepackage{pgfplots}
\usepackage{marvosym}
\usepackage{color}
\def\equationautorefname{}
\floatname{algorithm}{Algorithm: }
\renewcommand{\algorithmicrequire}{\textbf{Input:}}
\renewcommand{\algorithmicensure}{\textbf{Output:}}

\SetWatermarkText{}
\SetWatermarkLightness{0.92}
\SetWatermarkScale{0.49}
\pagestyle{fancy}
\numberwithin{equation}{section}
\CTEXoptions[today=old]
\captionsetup[figure]{name={Fig.}}
\title{\heiti \textbf{EI313\ Homework 1}}
\author{\kaishu 杨凯翔 519030910240, \href{mailto: flying_feixiang@sjtu.edu.cn}{\underline{flying\_feixiang@sjtu.edu.cn}}}
\bibliographystyle{plain}
\begin{document}
\maketitle

\thispagestyle{fancy}
\fancyhead{}
\lhead{\bfseries SJTU EI313, Autumn, 2021}
\rhead{\bfseries 工科创作业1}
\lfoot{}
\cfoot{\zihao{5}\copyright \ 2021 Kaixiang Yang. All Rights Reserved.}
\rfoot{\thepage}
\renewcommand{\headrulewidth}{0.4pt}
\renewcommand{\footrulewidth}{0.4pt}

\zihao{4}
\heiti
\section{Work}
In this homework, we need to install a simple vm which is different from our hostOS, and my hostOS is Win10, so I try to install ubuntu-20.10 vm by VMware.

First, SJTU has provided us with formal VMware Workstation and I have registered and got serial number in the last semester.

We also need *.iso file to install a vm in VMware, and we can acquire that from official website
\url{https://ubuntu.com/download/desktop}, or other mirror sites.
\begin{figure}[H]
    \centering
    \includegraphics[scale=1.0]{6.png}
    \caption{*.iso file}
\end{figure}
Then I create a vm and install ubuntu-20.10 as follows:
\begin{figure}[H]
    \centering
    \subfigure[pic1.]{
    \begin{minipage}[t]{0.25\linewidth}
    \centering
    \includegraphics[width=1.5in]{2.png}
    %\caption{fig1}
    \end{minipage}%
    }%
    \subfigure[pic2.]{
    \begin{minipage}[t]{0.25\linewidth}
    \centering
    \includegraphics[width=1.5in]{3.png}
    %\caption{fig2}
    \end{minipage}%
    }%
    \subfigure[pic3.]{
    \begin{minipage}[t]{0.25\linewidth}
    \centering
    \includegraphics[width=1.5in]{4.png}
    %\caption{fig2}
    \end{minipage}
    }%
    \subfigure[pic4.]{
    \begin{minipage}[t]{0.25\linewidth}
    \centering
    \includegraphics[width=1.5in]{5.png}
    %\caption{fig2}
    \end{minipage}
    }%
    \centering
    \caption{Procedure}
\end{figure}

And then just follow instructions to install OS.
Finally, I made it.
\begin{figure}[H]
    \centering
    \includegraphics[scale=0.4]{1.png}
    \caption{ubuntu-20.10 vm}
\end{figure}

\section{Afterwards}
Because I usually use WSL(ubuntu18.04) in my study to compile and run C/CPP program, I have been familiar with ubuntu terminal to some degree.
It will help for the next homework. 

And I will install a second OS in my host, using that iso file, to work on the following homeworks conveniently, and can also learn more knowledge from a linux system.

\end{document} 